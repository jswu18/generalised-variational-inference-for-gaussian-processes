\documentclass[twoside,11pt]{article}

\usepackage{blindtext}

% Any additional packages needed should be included after jmlr2e.
% Note that jmlr2e.sty includes epsfig, amssymb, natbib and graphicx,
% and defines many common macros, such as 'proof' and 'example'.
%
% It also sets the bibliographystyle to plainnat; for more information on
% natbib citation styles, see the natbib documentation, a copy of which
% is archived at http://www.jmlr.org/format/natbib.pdf

% Available options for package jmlr2e are:
%
%   - abbrvbib : use abbrvnat for the bibliography style
%   - nohyperref : do not load the hyperref package
%   - preprint : remove JMLR specific information from the template,
%         useful for example for posting to preprint servers.
%
% Example of using the package with custom options:
%
% \usepackage[abbrvbib, preprint]{jmlr2e}

\usepackage{paper}

% Definitions of handy macros can go here

\newcommand{\dataset}{{\cal D}}
\newcommand{\fracpartial}[2]{\frac{\partial #1}{\partial  #2}}

% Heading arguments are {volume}{year}{pages}{date submitted}{date published}{paper id}{author-full-names}

\usepackage{lastpage}
\usepackage{amsmath}
%\jmlrheading{23}{2022}{1-\pageref{LastPage}}{1/21; Revised 5/22}{9/22}{21-0000}{Author One and Author Two}

% Short headings should be running head and authors last names

\ShortHeadings{GVI in FS for Image Data}{Notes}
\firstpageno{1}

\begin{document}

\title{Generalised Variational Inference in Function Spaces for Image Data Notes}

% \author{\name Jian Shu (James) Wu \email jian.wu.22@ucl.ac.uk \\
%        \addr Department of Computer Science\\
%        University College London\\
%       Seattle, WA 98195-4322, USA
%       \AND
%       \name Author Two \email two@cs.berkeley.edu \\
%       \addr Division of Computer Science\\
%       University of California\\
%       Berkeley, CA 94720-1776, USA
% }

% \editor{}

\maketitle

%\begin{abstract}%   <- trailing '%' for backward compatibility of .sty file
%\blindtext
%\end{abstract}

%\begin{keywords}
%  keyword one, keyword two, keyword three
%\end{keywords}

\section{Gaussian Processes for Classification}\label{sec:gaussian-processes-for-classification}

Notes taken from chapter 4 from \cite{matthews2017scalable}.

For Gaussian process regression (GPR), a class of models is defined:
\[f \sim \mathcal{GP}(0, K(\theta))\]
where $f: X \rightarrow \mathbb{R}$, mapping to the set of real numbers $\mathbb{R}$ and $K$ is the covariance function $K: X \times X \rightarrow \mathbb{R}$ parameterised by $\theta$.

For binary Gaussian process classification (GPC), a mapping is defined:
\[g: \mathbb{R} \rightarrow [0, 1]\]
transforming a value on the real line to the unit interval to represent a probability.
A bernoulli random variable $\mathcal{B}$ can be defined such that:
\[f_c \sim \mathcal{B}(g(f))\]
where $f_c: X \rightarrow \{0, 1\}$, the desired binary classifier.

For multiclass classification of $J$ different classes, models are defined:
\[f^{(j)} \sim \mathcal{GP}(0, K(\theta^{(j)}))\]
where $j=1, \dots, J$, defining $J$ i.i.d. Gaussian processes.
Concatenating $\mathbf{f} = [f_1 \cdots f_J]^T$, the classification operation can be defined:
\[\mathbf{f}_c \sim Cat (\mathcal{S}(\mathbf{f}))\]
where $\mathbf{f}_c: X \rightarrow \{0, \dots, J\}$, the desired multiclass classifier. $\mathcal{S}: \mathbb{R}^J \rightarrow \Delta(J)$, a mapping from a $J$ dimensional real vector to a $J$ dimensional probability simplex. $Cat$ is the categorical distribution (generalisation of Bernoulli distribution for Categorical Data).

There are different possible choices for $\mathcal{S}$.
The multiclass generalisation of the logit likelihood:
\[\mathcal{S}_{softmax}(\mathbf{f})_i = \frac{\exp(f^{(i)})}{\sum_{j=1}^{J} \exp(f^{(j)})}\]

The robust max function:
\[\mathcal{S}_{robust}^{(\epsilon)}(\mathbf{f})_i = \begin{cases}
      1-\epsilon, &  \text{if } i = \arg \max(\mathbf{f}) \\
      \epsilon, & \text{otherwise} \\
   \end{cases}\]
taking class label of the maximum value with probability of $1-\epsilon$ and probability $\epsilon$ of picking one of the other classes uniformly at random, where $\epsilon$ is chosen.
This formulation provides robustness to outliers, as it only considers the ranking of the GPR models for each class.

A benefit of the robust max function is that the variational expectation is analytically tractable with respect to the normal CDF ($q(\mathbf{f}) = \mathcal{N}(\mu, C), \mathbf{f} \in \mathbb{R}^J$) and one dimensional quadrature ($\mathcal{S}_{robust}^{(\epsilon)}(\mathbf{f})_i \in \mathbb{R}$):
\[\int_{\mathbb{R}^J} q(\mathbf{f}) \log(\mathcal{S}_{robust}^{(\epsilon)}(\mathbf{f})_y) d\textbf{f} = \log(1-\epsilon) S + \log\left(\frac{\epsilon}{J-1}\right)(1-S)\]
where $S$ is the probability that the function value corresponding to observed class $y$ is larger than the other function values at that point:
\[S = \mathbb{E}_{\mathbf{f}^{(y)} \sim \mathcal{N}(\mathbf{f}^{(y)} | \mu^{(y)}, C^{(y)})} \left[\prod_{i \neq y} \phi \left(\frac{\textbf{f}^{(y)}-\mu^{(i)}}{\sqrt{C^{(i)}}} \right)\right]\]
where $\phi$ is the standard normal CDF. This one dimensional integral can be evaluated using Gauss-Hermite quadrature.


% Acknowledgements and Disclosure of Funding should go at the end, before appendices and references

%\acks{All acknowledgements go at the end of the paper before appendices and references.
%Moreover, you are required to declare funding (financial activities supporting the
%submitted work) and competing interests (related financial activities outside the submitted work).
%More information about this disclosure can be found on the JMLR website.}

% Manual newpage inserted to improve layout of sample file - not
% needed in general before appendices/bibliography.

\newpage

%\appendix
%\section*{Appendix A}

\bibliography{notes}

\end{document}